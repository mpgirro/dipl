%\documentclass[handout]{beamer} % use this to disable \pause commands
\documentclass{beamer}
%
% Choose how your presentation looks.
%
% For more themes, color themes and font themes, see:
% http://deic.uab.es/~iblanes/beamer_gallery/index_by_theme.html
%
\mode<presentation>
{
  \usetheme{default}      % or try Darmstadt, Madrid, Warsaw, ...
  \usecolortheme{default} % or try albatross, beaver, crane, ...
  \usefonttheme{default}  % or try serif, structurebold, ...
  \setbeamertemplate{navigation symbols}{}
  \setbeamertemplate{caption}[numbered]
} 

\usepackage[english]{babel}
\usepackage[utf8x]{inputenc}

\newcommand\mydots{\hbox to 1em{.\hss.\hss.}}

\title[Your Short Title]{Concurrent Programming with\\Actors and Microservices}
\author{Maximilian Irro}
\date{Diplomprüfung\\XX.11.2018}

\begin{document}

\begin{frame}
  \titlepage
\end{frame}

% Uncomment these lines for an automatically generated outline.
%\begin{frame}{Outline}
%  \tableofcontents
%\end{frame}

\section{Concurrency}

% TODO hier sollte vielleicht noch ein anderer Slide vorher stehen

\begin{frame}{Forms of Concurrency}

\begin{itemize}
  \item Concurrent execution
  \item Parallel execution
  \item Distributed execution
\end{itemize}

\end{frame}


\section{Actor Model}

\begin{frame}{Actors}

\begin{itemize}
  \item TODO
\end{itemize}

\end{frame}

\section{Microservices Paradigm}

\begin{frame}{Microservices}

\begin{itemize}
  \item TODO
\end{itemize}

\end{frame}


\end{document}

