\documentclass{beamer}
%
% Choose how your presentation looks.
%
% For more themes, color themes and font themes, see:
% http://deic.uab.es/~iblanes/beamer_gallery/index_by_theme.html
%
\mode<presentation>
{
  \usetheme{default}      % or try Darmstadt, Madrid, Warsaw, ...
  \usecolortheme{default} % or try albatross, beaver, crane, ...
  \usefonttheme{default}  % or try serif, structurebold, ...
  \setbeamertemplate{navigation symbols}{}
  \setbeamertemplate{caption}[numbered]
} 

\usepackage[english]{babel}
\usepackage[utf8x]{inputenc}

\newcommand\mydots{\hbox to 1em{.\hss.\hss.}}

\title[Your Short Title]{Concurrent Programming\\and the Microservice Architecture Style}
\author{Maximilian Irro}
\date{Seminar für DiplomandInnen\\4.12.2017}

\begin{document}

\begin{frame}
  \titlepage
\end{frame}

% Uncomment these lines for an automatically generated outline.
%\begin{frame}{Outline}
%  \tableofcontents
%\end{frame}

\section{Timeline of the Microservice concept}

\begin{frame}{Timeline of the Microservice concept}

\begin{itemize}
  \item 2012 (?) first mention of the term “Microservice”
  \item 2014 Whitepaper by Fowler and Lewis
  \item 2015 Academia gets interested
  \item 2016, 2017 Explosion of publications
\end{itemize}

\end{frame}

\section{Definitions}

\begin{frame}{Definitions}

\pause

"A \textbf{microservice} (MS) is a cohesive, independent process interacting via messages"

\pause
\vskip 1cm

"A \textbf{microservice architecture} (MSA) is a distributed application where all its components are microservices"

\vskip 2cm

\tiny{[Dragoni, Nicola, et al. "Microservices: yesterday, today, and tomorrow."Present and Ulterior Software Engineering. Springer, Cham, 2017. 195-216.]}

\end{frame}

\section{Characteristics of Microservices}

\begin{frame}{Characteristics of Microservices (1)}

  \begin{itemize}
    \item Services as building blocks
    \item Focus on business capability
    \item Multiple executable artifacts
    \item Communication via messages
    \item Lightweight communication (e.g. REST, “dumb” message broker)
    \item No-shared state, only accessible via API
    \item High cohesion, loose coupling
    \item etc…
  \end{itemize}

\end{frame}

\begin{frame}{Characteristics of Microservices (2)}
    \begin{columns}[T] % the "c" option specifies center vertical alignment
    \begin{column}[T]{5.5cm} % each column can also be its own environment
    	\textbf{Microservices}
    	\begin{itemize}
        \item Services are building blocks
        \item Identity (PID, network-address), state, behavior
        \item Communication via messages
        \item Interfaces, design by contract
        \item Encapsulation, data-hiding
        \item High cohesion, loose coupling
        \item …
      	\end{itemize}
    \end{column}
    \pause
    \begin{column}[T]{5.5cm} % alternative top-align that's better for graphics
		\textbf{Objects}
    	\begin{itemize}
        \item Services are building blocks
        \item Identity (memory-address), state, behavior
        \item Communication via messages
        \item Interfaces, design by contract
        \item Encapsulation, data-hiding
        \item High cohesion, loose coupling
        \item …
      	\end{itemize}
    \end{column}
    \end{columns}
\end{frame}

\begin{frame}{Characteristics of Microservices (cont.)}

  \begin{itemize}
    \item MSA consists of multiple processes 
    \item Message passing is inter-process communication
    \item Application is a distributed system
  \end{itemize}

\end{frame}

\begin{frame}{Are MS just Distributed Objects?}

  \begin{itemize}
    \item Distributed Objects based on idea of 

    \begin{itemize}
      \item Putting objects into processes
      \item Transparent in-process/remote communication
    \end{itemize}
    
    \pause
    
    \item MS make remote method invocation (e.g. REST call) explicit
    \begin{itemize}
      \item Different API granularity
    \end{itemize}

  \end{itemize}
  
  \vskip 2cm

\tiny{[Waldo, Jim, et al. "A Note on Distributed Computing."International Workshop on Mobile Object Systems. Springer, Berlin, Heidelberg, 1996.]}

\end{frame}

\begin{frame}{Concurrency and Microservices}

  \begin{itemize}
    \item Threads vs Processes
	\pause 
    \item Concurrency Models often look similar to Distributed Systems Architecture
	\pause
    \item ``[\mydots] expect to see a return of distributed programming with increasingly fine-grained distributed systems that will make systems look more and more like classic concurrent system''

  \end{itemize}
  
  \vskip 2cm

\tiny{[Jan Stenberg. „Concurrent and Distributed Programming in the Future“. 2017. https://www.infoq.com/news/2017/03/distributed-programming-qcon]}

\end{frame}

\begin{frame}{Special Interest Aspects of MSA}

  \begin{itemize}
    \item Geographically distributed code
    \item Explicit parallelization (via deployment)
    \item Fault tolerance, resilience, self-healing
    \item Scalability
    \item Independent replacement and upgradability
    \item Decentralized data management (no shared memory)
    \item Polyglot programming, polyglot persistence
  \end{itemize}

\end{frame}

\begin{frame}{Roadmap}

  \begin{itemize}
    \item Conceptual similarities between MS and OO, SOA, etc.
    \item Identify concurrency model(s) resembling MSA
    \item Practically demonstrate and compare with prototypical scenario
    \item Argue if ``sufficiently concurrent'' programming languages need MSA? 
  \end{itemize}

\end{frame}

\begin{frame}{Actors (Akka implementation)}

  \begin{itemize}
    \item Objects encapsulate state and behavior
    \item (Asynchronous) message passing, immutable messages
    \item Isolated mutable state, no global state
    \item Distributed by default
    \item Fault tolerance (supervision hierarchy)
  \end{itemize}

\end{frame}

\end{document}

